\newpage

\section{Remerciements}\label{remerciements}

\bigskip

Je tiens tout d'abord à remercier Benjamin Tierny, Robin Komiwes et
Julien Vanden Torren de m'avoir acceuilli chez Dernier cri pour mon
stage.

\bigskip

Je remercie également mon suiveur Harry Claisse pour son aide et son
accompagnement, ainsi que les enseignants de l'Université technologique
de Compiègne.

\bigskip

Merci également à toute l'équipe de Dernier cri pour avoir rendu mon
stage si enrichissant et agréable et pour m'avoir accueuilli à bras
ouverts.

\bigskip

J'aimerai finalement remercier mes parents et mon frère, qui m'ont
permis de faire ses études et m'ont soutenu durant ce stage.

\newpage

\iffalse
\# Résumé et abstract

\bigskip

\subsection{Résumé}\label{ruxe9sumuxe9}

\newpage

\subsection{Abstract}\label{abstract}

\newpage

\fi

\section{Introduction}\label{introduction}

\bigskip

Dans le cadre de mon stage de TN09 lors de ma quatrième année
d'ingénieur à l'Université technologique de Compiègne (UTC), j'ai
effectué un stage de six mois chez Dernier cri.

\bigskip

Dernier Cri est une Start-Up créé en 2011 spécialisée dans l'innovation
numérique. L'équipe est en charge du développement, du déploiement et de
la maintenance d'applications pour le compte de plusieurs clients.

\bigskip

Ma mission a été d'intégrer l'équipe de développement pour aider dans la
création de plusieurs applications web.

\bigskip

J'ai pu lors de ce stage intégrer une équipe dynamique et pro-active.
J'ai notamment pu prendre part à de nombreuses présentations internes
sur différentes technologies, à l'écriture d'articles de blog. J'ai
également pu participer activement à la relation client lors de mes
projets.

\bigskip

Dans ce rapport je vais vous présenter tout d'abord Dernier cri,
l'entreprise qui m'a acceuilli. Il me tient à coeur de vous exposer les
techniques de travail et les différents moyens de s'exprimer qu'elle
offre à ses équipes.

\bigskip

Je vous exposerai ensuite ma mission au sein de l'entreprise. Je
détaillerai les technologies que j'ai utilisé mais aussi la gestion de
projet ainsi que la relation client, qui sont des éléments clés dans la
bonne conduite d'un projet.

\bigskip

Finalement, il me tient à coeur de présenter la communauté de Lille et
ses particularités. En effet, le dynamisme de la communauté et la
diversité des événements proposés m'ont beaucoup aidé à m'intégrer et à
approfondir mon projet professionnel.

\newpage

\section{Dernier cri}\label{dernier-cri}

\bigskip

\subsection{Histoire}\label{histoire}

\bigskip

En 2011 Robin Komiwes et Benjamin Tierny créent Nectify dans le but de
développer Fresc, un outil de partage d'avis sur des visuels. Bien que
cet outil connait un succès certain avec aujourd'hui plus de 300
sociétés utilisant Fresc à travers des milliers de projets, la
rentabilité du projet n'est pas suffisante.

\bigskip

Nectify choisit alors de compléter ses revenus par de la prestation de
services centrée sur l'innovation.

\bigskip

Début 2014, la majeure partie du chiffre d'affaire de Nectify était dû
aux activités de prestations de services, Fresc ne représentant qu'une
part marginale.

\bigskip

Devenant donc une agence spécialisée dans l'innovation digitale, Nectify
choisit de créer sa propre image, distincte de Fresc. C'est dans ce
mouvement que la société est devenue \textbf{Dernier cri}.

\bigskip

Aujourd'hui Dernier cri est une agence web qui met un point d'honneur à
proposer à ses clients une solution complète adaptée à leur
problèmatique spécifique. De la conception à la réalisation,
l'entreprise accompagne ses clients de A à Z pour aboutir à un produit
au plus proche des besoins de ceux-ci. Cela permet aux développeurs
d'opérer dans différents domaines d'activités et d'avoir une vue globale
du développement de produit.

\bigskip

\subsection{Secteur d'activité}\label{secteur-dactivituxe9}

\bigskip

Le secteur de l'informatique est aujourd'hui est en pleine expension et
ne connait pas la crise. Pour donner un ordre d'idée, le marché de la
programmation et des services informatiques embauche près de 400 000
personnes dans ses 21 000 ESN (Entreprises de services du numérique) et
ne cesse de croître depuis ces cinq dernières années.

\bigskip

Dernier cri est une \emph{startup} spécialisée dans l'innovation
digitale. Elle se démarque notamment des autres agences web en proposant
plusieurs services, dont évidemment la création d'applications
web/mobile, mais aussi l'application de la recherche fondamentale en
apprentissage automatique et traitement de grosses données, afin
d'augmenter la capacité d'innovation d'entreprises tiers. Cela se
traduit notamment par de l'exploration de connaissances à partir de
données pour des magasins, par exemple, ou la création d'audit.

\bigskip

C'est cette diversité qui permet à Dernier cri de se démarquer dans
l'univers du web à Lille, où de nombreuse agence se partage le marché.

\bigskip

\subsection{Organisation}\label{organisation}

\bigskip

Dernier cri est une entreprise de petite taille et profite de cela pour
mettre en place une organisation souple qui offre l'opportunité à tous
de s'impliquer dans les projets, que ce soit des projets pour des
clients ou des projets interne.

\bigskip

Au sommet de l'organisation de l'entreprise se trouve les deux
fondateurs ainsi que le troisième associé. Le directeur général de
l'entreprise (CTO) est Benjamin Tierny, tandis que Robin Komiwes est le
directeur de la technologie (CTO). Le troisième associé, Julien Vanden
Torren, est Account Manager, c'est à dire le lien entre l'entreprise et
les clients.

\bigskip

L'entreprise compte aujourd'hui 18 employés mais se trouve dans une
phase d'expension avec de nouvelles embauches en perspective, grace à de
nombreux nouveaux projets.

\bigskip

L'entreprise est également constituée d'une chef de projet, Laetitia
Cocusse, qui s'occupe de superviser la plupart des projets.

\bigskip

L'équipe de développeur est constituée notament d'un devops, c'est à
dire d'une personne possédant à la fois les compétences d'un développeur
et d'un ingénieur système (Jean-Serge Monbailly), d'un Data Scientist
(Antonin Carette) qui travaille sur les projets de big data.

\bigskip

La plupart des développeurs travaille sur plusieurs projets en même
temps, selon les besoins de l'entreprise et les compétences de chacun.
Des équipes de 2/3 développeurs sont crées, mélangeant les compétences
front et back, de développement et de système.

\bigskip

\subsubsection{Github et code review}\label{github-et-code-review}

\bigskip

L'entreprise utilise principalement la plateforme Github comme service
web d'hébergement et donc le logiciel de gestion de versions Git. Github
permet aux développeurs de travailler à plusieurs sur le même projet, de
résoudre rapidement des conflits dus à la modification d'un même
document par plusieurs personnes, de gérer différentes versions du
projet et beaucoup d'autre problématiques.

\bigskip

Une nouvelle version de Github propose une section \textbf{Projet}
permettant de gérer les \textbf{issues}, c'est à dire les tâches. Cette
section permet notamment de séparer les tâches en plusieures colonnes,
par exemple : \emph{A faire}, \emph{En cours}, \emph{Terminé}. Il est
aussi possible d'attribuer les tâches à un contributeur, ou encore de
leur attribuer des labels tel que \emph{Urgent}, \emph{Bug} ou encore
une estimation de temps quand à la réalisation de la che.

\bigskip

Dernier cri utilisait jusque là un outil similaire. Il a été décidé
d'utiliser la section \textbf{Projet} de Github pour les nouveaux
projets, notamment ceux sur lesquels j'ai été affectée.

\bigskip

Dernier cri utilise ces outils mis à disposition par Github pour mettre
en place un processus de vérification de la qualités du code et
d'entraide. Chaque developpeur, une fois une tâche terminée, propose une
\emph{Pull request}, c'est à dire demande à fusionner sa version du
projet, modifiée pour résoudre la tâche, avec la version principale,
stable. Il demande ensuite à ses collégues ayant des compétences dans le
langage utilisé de relire et de commenter cette \emph{Pull request}.

\bigskip

C'est l'occasion pour les développeurs d'avoir l'avis de leurs collégues
sur leur style d'écriture et leur façon de coder, ce qui permet souvent
de découvrir de nouvelles méthodes et d'argumenter sur les meilleurs
techniques à utiliser. Dernier cri utilise ce système pour garantir une
certaine qualité du code ainsi qu'un style d'écriture homogéne.

\bigskip

\subsubsection{Talk interne}\label{talk-interne}

\bigskip

Dans l'optique d'un partage du savoir dans l'entreprise, les employés
sont invités à faire des \emph{talks}, c'est à dire de petites
présentations. Celles-ci ont pour objectif premier de présenter certains
enjeux et solutions techniques en rapport avec la réalisation d'un
projet, le résultat d'une veille, ou tout simplement un sujet qui les
intéresse. Les \emph{talks} techniques permettent aux membres d'une
équipe de partager entre eux leurs expériences et leurs passions.

\bigskip

Ces présentations ont deux principaux avantages. Tout d'abord elles
permettent aux spectateurs de la présentation d'apprendre de nouvelles
choses. C'est l'occasion de découvrir un sujet dont on ne se doutait pas
de l'intéret ou bien de l'existence. De plus, la présentation est
travaillée et structurée, car une personne a déjà fait l'effort de trier
les informations et de les présenter de la manière la plus lisible,
claire, ou ludique. C'est ainsi souvent beaucoup plus facile d'aborder
un sujet avec ces talks, que de partir à la pêche aux informations sur
internet.

\bigskip

Le second avantage concerne l'organisateur de la présentation. Tout
d'abord, sensibiliser ses coéquipiers à des problèmes peut être un
véritable gain de temps pour le futur. En effet, apporter ses
connaissances à l'équipe en entier permet d'éviter de devoir aller aider
ou expliquer individuellement les même choses.

\bigskip

De plus, le rédacteur apprend beaucoup car, pour la création de la
présentation, il doit devenir un expert sur le sujet qu'il va couvrir.
Il doit appronfondir assez le sujet pour être capable, d'une part d'être
clair dans sa présentation, d'autre part de répondre aux éventuelles
questions.

\bigskip

Pour lui c'est aussi une occasion de travailler sur ces compétences
d'orateur. Cela peut évidemment lui être utile pour son travail, sa vie
de tout les jours, mais surtout un techtalk interne peut faire office
d'incubation pour une présentation dans un autre contexte. En effet, il
existe à Lille de nombreux évènements où il est possible de faire des
présentations, ce qui est très intéréssant mais aussi peut être assez
intimidant. Le présentateur peut tester son talk en interne avant d'en
faire partager la communauté.

\bigskip

Finalement, présenter un sujet issu d'une veille technologique permet,
au-delà de faire découvrir aux autres un sujet potentiellement
intéressant, de certifier l'intérêt du sujet. En effet, beaucoup de
sujets qui au premier coup d'oeil semblaient prometteur, peuvent avec un
peu de recherche se révéler creux et sans intêret.

\bigskip

Les talks internes sont une très bonne manière de partager des
connaissances de manière ludique. Durant mon stage j'ai eu l'occasion
d'assister à des présentations sur de nombreuses technologies telles que
React, Docker, Rust, MJML\ldots{} Mais aussi sur des sujets variés comme
le Growth hacking (Ensemble de techniques de marketing permettant
d'accélérer rapidement et significativement la croissance d'une
start-up.), ou encore la fabrication d'un audit.. Ces présentations sont
publiées sur la
\href{https://www.youtube.com/channel/UCDfdBlzldhg_PEu3xZTPsHg}{chaîne
youtube Dernier cri}. Cela m'a permis de découvrir de nouveaux sujets et
de mieux comprendre les discutions, et les problématiques que
rencontraient mes collégues.

\bigskip

\subsubsection{Article de blog}\label{article-de-blog}

\bigskip

Dernier cri posséde également un
\href{http://derniercri.io/tech-blog}{blog technique} alimenté par les
développeurs de l'équipe. Les objectifs sont notablement les mêmes que
pour les présentations internes. Mais cela permet aussi à Dernier cri de
rayonner et montrer ses compétences techniques et son esprit d'analyse.

\bigskip

Le blog technique permet à Dernier cri de se positionner en tant
qu'expert. Publier des articles blog est excellent pour la réputation.
En délivrant des conseils professionnels utiles, il prouve leur maîtrise
de leur activité.

\bigskip

Cela permet également d'améliorer la visibilité de l'entreprise en
exposant son savoir et savoir-faire à une large communauté
professionnelle. Un blog est également utile pour tenir informé sa
communauté : il permet de rester en contact permanent avec ses prospects
et de les tenir informé de toute l'actualité de son entreprise, comme un
nouveau produit, une nouvelle technologie maitrisée \ldots{}

\bigskip

Pour le rédacteur de l'article, cela a également de nombreux avantages.
Par exemple, cela lui permet de travailler sa rédaction et son
argumentation, de partager avec d'autres développeurs à travers les
commentaires, de gagner en visibilité et en réputation, ou encore de se
forcer à rester à la page, pour être exhaustif et à jour dans ses
références.

\bigskip

J'ai moi-même eu l'occasion lors de mon stage d'écrire un article pour
le blog technique de Dernier cri. Dans cet article je décris les
technologies que j'ai utilisé lors de mon stage, React et Reduc, ainsi
que cinq outils que j'ai découvert et qui m'ont facilité la vie pour
utiliser ses technologies. Ce fut une expérience très enrichissante.
Cela m'a poussé à enrichir mes connaissances dans le domaine, de
clarifier les concepts pour savoir les expliquer à tous et de travailler
ma qualité de rédaction.

\newpage

\section{Cadre du stage}\label{cadre-du-stage}

\bigskip

Durant mon stage j'ai rejoint l'équipe de développeurs de l'entreprise
et j'ai pu participer au développement de deux applications. Ces deux
projets s'appuyaient sur du React, une bibliothèque JavaScript libre
développée par Facebook depuis 2013. N'ayant jamais utilisé cette
bibliothèque, j'ai donc du tout d'abord me former.

\bigskip

Le premier projet auquel j'ai participé se nomme Photolix. C'est un site
internet de développement de photos, avec pour objectifs de toucher un
large public et de limiter au maximum le temps d'attente du client en
envoyant les photos au serveur dès leur sélection.

\bigskip

Le second projet est FinFrog, un site proposant des prêts financés par
des particuliers. Ce projet était déjà assez avancé à mon arrivé. Le
client possédait un site en ligne, mais souhaité changer l'apparence et
ajouter des fonctionnalités, ce pourquoi il a fait appel à Dernier cri.

\bigskip

\subsection{Formation}\label{formation}

\bigskip

Lors de mon arrivée chez Dernier cri, j'ai eu l'occasion de me former
sur Javascript ES6, React ainsi que Redux, car l'entreprise prévoyait de
me mettre sur des projets utilisant ces technologies.

\bigskip

La popularité de ces trois technologies est en forte hausse, et de plus
en plus demandée et utilisée pour la création d'applications web. C'est
pourquoi ce fut une véritable chance d'apprendre ces langages lors de
mon stage.

\bigskip

Ma formation s'est faite à partir du site
\href{https://www.codeschool.com/}{Code school}, disposant de cours en
vidéos ainsi que d'exercices interactifs. Par la suite j'ai également
pus compter sur Fabien Gavory, un développeur de Dernier cri, pour
m'aider et m'expliquer certains concepts difficiles.

\bigskip

\subsubsection{Javascript ES6}\label{javascript-es6}

\bigskip

Le Javascript est un langage de programmation de scripts incontournable
du web. Si à sa création il servait principalement à la réalisation
d'animation, il est aujourd'hui au centre des applications. Le javacript
sert maintenant à controler presque la totalité de l'application web.
Cependant, ce langage n'étant pas prévu pour une telle complexité, il en
résulte une syntaxe complexe et lourde. C'est dans ce contexte qu'une
mise à jour du langage s'est imposée.

\bigskip

ES6 (ECMAScript Edition 6 ou encore ES2015) a été publiée en juin 2015.
Il ajoute un ensemble de normes à celles déjà présentes, pour apporter
de nouvelles fonctionnalités qui permettent d'alléger le code, de le
structurer, et de le rendre notamment plus maintenable, tout en restant
compatible avec le code existant.

\bigskip

ES6 n'est pas encore totalement supporté par les navigateurs, il est
donc utile d'utiliser un transcompilateur vers ES5, comme Babel.js.

\bigskip

L'apprentissage de ES6 a était primordial pour mon stage : les nouvelles
normes rendent vraiment le code plus facile à lire et à écrire. J'ai
ainsi dès le début de mon stage, pus prendre de bonnes habitudes quant
au style de mon code.

\bigskip

Cela m'a également permi de me familiariser avec les normes ECMAScript
et de me persuader de la nécessité de rester attentive aux différentes
actualités et évolutions des langages. En effet, dans le milieu de
l'informatique, les normes et bonnes pratiques sont très changeantes et
il est donc important de rester attentif à l'actualité.

\bigskip

J'ai eu l'occasion durant mon stage de travailler sur un projet
JavaScript n'utilisant pas ES6 et j'ai eu de grande difficultés à me
passer des facilités d'écritures. Sans ES6, le code est beaucoup plus
long et laborieux à lire et à écrire.

\bigskip

\subsubsection{React}\label{react}

\bigskip

Développé depuis 2013 par Facebook, \emph{React} est une bibliothèque
JavaScript déclarative, efficace et flexible pour la création
d'interfaces utilisateurs. Cette bibliothèque s'est démarqué notamment
par ses performances.

\bigskip

Elle est aujourd'hui utilisée par de nombreuses entreprises telles que
Netflix, Yahoo, Airbnb ou encore Sony.

\bigskip

Une des particularités de \emph{React} est de découper l'application en
composants, dépendant d'un état. Lors du changement de l'état d'un
composant, \emph{React} génère les changements en \emph{HTML} pour les
répercutés sur la page.

\bigskip

Cette bibliothèque est aujourd'hui en expansion. Elle a un succès
certain auprès de la communautés des développeurs web, et de nombreux
outils se développent autours. C'est donc un avantage d'avoir pu
apprendre \emph{React} lors de mon stage, puis d'avoir mis en pratique
ces connaissances lors des deux projets que j'ai effectué.

\bigskip

\subsubsection{Redux}\label{redux}

\bigskip

React, même s'il n'impose pas de bibliothèque pour les donnés et la
communication des composants, offre une approche nommée \emph{Flux} très
intéressante.

\bigskip

En plus de \emph{React}, Facebook a fourni une architecture appelée
\emph{Flux} pour la gestion des donnés et la communication des
composants. Cette architecture promet un flot unilatéral des données
pour que le développeur puisse facilement suivre le trajet des données
d'un événement et ses conséquences.

\bigskip

\emph{Redux} est une des implémentations de \emph{Flux} les plus
populaires, créée en Mai 2015 par Dan Abramov. Bien que reprenant les
concepts de \emph{Flux}, \emph{Redux} les simplifie en utilisant des
concepts liés à la programmation fonctionnelle.

\bigskip

La mise en place de \emph{Redux}, ou de \emph{Flux} en général peut
sembler dans un premier temps laborieuse, car elle impose de nombreux
fichiers, de nombreux composants différents et une façon de penser qui
peut désarçonner au début. Mais une fois les bases mises en place, cette
architecture a l'avantage d'offrir une bonne lisibilité quant aux
changements d'état. Il est possible de se retrouver dans le code très
rapidement et d'ajouter des composants sans difficultés puisque
\emph{React} et \emph{Flux} sont conçus pour être modulable. Flux trouve
donc son utilité dans les grandes applications.

\bigskip

Bien que \emph{Redux} soit la seule implémentation de \emph{Flux} que
j'ai eu l'occasion d'utiliser, je pense qu'il s'agit d'une des
meilleurs. Elle permet d'appréhender certain concept de la programmation
fonctionnel, est simple à utiliser et assez populaire pour offrir un
support en cas de problème.

\bigskip

\subsection{Gestion de projet}\label{gestion-de-projet}

\bigskip

Le processus de développement chez Dernier cri se passe généralement
comme suit. Tout d'abord notre chef de projet, Laetitia Cocusse, discute
avec le client pour comprendre ses besoins et ses attentes. Selon les
besoins, elle organise une réunion avec le(s) developpeur(s) et le
client pour clarifier certains points ou pour discuter des différents
approches du problèmes possible. C'est l'occasion pour le développeur de
proposer des solutions, peut être un peu différentes de celles imaginées
par le client, ou encore de dire son avis sur les prochaines tâches.

\bigskip

Ensuite les taches sont estimées par un développeur. Il estime le temps
qu'il pense passer sur le problème, en prenant en compte l'exploration
de l'existant, le développement en lui-même, les tests et les possibles
retours. Cette estimation devra ensuite être validée par le client.

\bigskip

Une fois la tache validée, une \emph{issue} est créé dans Github avec
une description, l'estimation et le développeur à qui est attribuée la
tache. Cela permet à chacun d'avoir une vue à tout moment de l'évolution
du projet, des tâches en cours ou terminées. Il est également possible
de commenter chaque \emph{Issue} pour, par exemple, demander des
précisions, ou faire remonter une erreur.

\bigskip

Selon les priorités, le développeur choisit ou non l'ordre des tâches.
Pour protéger le projet actuel, stable, des nouvelles modifications tant
que celles-ci ne sont pas testées, le développeur crée une
\emph{branche} dans Github, c'est à dire une copie du projet qui
évoluera indépendamment de la branche principale. C'est sur cette
branche que seront faites les modifications destinées à implémenter la
nouvelle fonctionnalité.

\bigskip

C'est ensuite le moment de passer au développement à proprement parler.
Le développeur utilise un environnement de developpement sur son propre
ordinateur pour simuler l'environnement de production. Cela peut se
traduire par la création d'une base de données, ou la connexion à une
\emph{API} spéciale. Il faut faire attention à ce que les données
utilisée lors de la phase de développement n'aient pas d'impact sur
celles de production.

\bigskip

Quand le développeur estime avoir terminer la tâche, il crée une
\emph{Pull request} sur Github (comme expliqué plus haut dans le
rapport). C'est alors le moment de prendre en compte les remarques et
conseils des autres membres de l'équipe, de corriger éventuellement des
éléments, pour s'assurer de la qualité du code avant de l'incorporer
dans le projet.

\bigskip

Une fois le développement de la fonctionnalité validé, la nouvelle
version du site est déployé en \emph{Staging}, c'est à dire une version
en ligne du site, qui a le même environnement que la production, mais
avec des fausses données. Ce site sert à tester les nouvelles
fonctionnalités avant de les pousser sur la production.

\bigskip

Laetitia fait une \emph{recette}, c'est à dire vérifie que la version de
\emph{staging} remplit bien le besoin exprimé par le client, et que la
nouvelle fonctionnalité n'a pas cassé autre chose. En cas de problème,
le développeur revient sur la tâche jusqu'à ce que tout soit réglé.

\bigskip

Une fois un lot de taches effectuées, il est décidé en accord avec le
client de pousser les modifications sur la production. Il faut alors
vérifier que la \emph{mise en prodcution} c'est bien passée : que le
site fonctionne toujours et que les nouvelles fonctionnalités sont bien
en place.

\newpage

\section{Mes réalisations}\label{mes-ruxe9alisations}

\bigskip

J'ai eu la chance de participer à plusieurs projets durant mon stage, de
façon plus ou moins importantes. Je vais vous présenter dans cette
partie les deux principaux projets sur lesquels je me suis investie.

\bigskip

J'ai aussi pu travailler sur d'autre projet en renfort sur de courtes
périodes, ainsi que développer un outils pour le site de Dernier cri.

\bigskip

\subsection{Photolix}\label{photolix}

\subsubsection{Présentation du projet}\label{pruxe9sentation-du-projet}

\bigskip

Dès mon arrivée dans l'entreprise j'ai été assigné à la réalisation
d'une application de développement photo. Le client possède un studio de
développement photo sur Lille, et souhaitait proposer à sa clientèle un
site simple et efficace.

\bigskip

Le principal objectif de ce projet était de télécharger les photos vers
le serveur au fur et à mesure de leur sélection, pour ainsi éviter le
temps d'attente du client à la fin de la saisie de ses informations.

\bigskip

Lors de mon arrivée sur le projet, un développeur de Dernier cri avait
déjà posé des bases. Les fonctions de recadrage et de compression de la
photo était notamment déjà écrite.

\bigskip

\subsubsection{Objectifs}\label{objectifs}

\bigskip

Les objectifs du projet Photolix étaient les suivants :

\begin{itemize}
\tightlist
\item
  mettre en place le \emph{design} fournit par le client, à partir des
  maquettes sur Zeplin;
\item
  gérer l'envoi des photos au serveur;
\item
  mettre en place la possibilité de modifier les photos (formats,
  orientation\ldots{});
\item
  pages de saisie des informations du client (adresses, informations de
  paiement) et page de remerciement;
\end{itemize}

\bigskip

\subsubsection{Outils utilisés}\label{outils-utilisuxe9s}

\bigskip

Lors de mon arrivée sur ce premier projet, j'ai du apprendre à utiliser
certains outils, tant au niveau de la gestion de projet, que du
développement en lui-même.

\bigskip

Tout d'abord, le projet utilise Github comme service web d'hébergement
et de gestion de développement, et par conséquent le logiciel de gestion
de versions Git. Bien qu'ayant déjà utilisé Git et Github lors de mon
DUT informatique, de projet personnel ou bien de projet à l'UTC, je ne
connaissais pas certaines fonctionnalités de Github utilisées par
l'entreprise, notamment le code review et l'onglet projet. (Voir plus
haut)

\bigskip

Le projet déjà existant utilisait npm comme gestionaire de paquets. npm
est le gestionnaire de paquets officiel de Node.js. , automatiquement
installé par défaut depuis la version 0.6.3 de Node.js. npm fonctionne
avec un terminal et gère les dépendances pour une application. Il permet
également d'installer des applications Node.js disponibles sur le dépôt
npm. Il offre également la possibilité de créer des scripts. C'est une
option vraiment pratique car grace à cela on peut construire et lancer
l'application en une commande.

\bigskip

Pour mon environnement de travail, j'ai aussi utilisé un Linter. Code
linting est un type d'analyse statique qui est fréquemment utilisé pour
trouver des modèles problématiques ou le code qui ne respecte pas
certaines directives de style. Il existe des linters de code pour la
plupart des langages de programmation, et les compilateurs incorporent
parfois le linting dans le processus de compilation.

\bigskip

J'ai personnellement utilisé ESLint, qui est un utilitaire JavaScript
open-source et libre créé à l'origine par Nicholas C. Zakas en Juin
2013. ESLint est écrit en utilisant Node.js pour fournir un
environnement d'exécution rapide et une installation facile via npm. La
principale raison pour laquelle ESLint a été créé était de permettre aux
développeurs de créer leurs propres règles de filtrage.

\bigskip

J'ai également pu utiliser le \emph{chatops} de Dernier cri, un outil
d'administration système via la conversation. Intégré au Slack de
l'entreprise, il permet à tout le personnel d'obtenir des informations
sur un serveur ou une application et d'effectuer des résolutions simples
en cas de panne. Concraitement, j'ai principalement utilisé le chatops
pour déployer mon application.

\bigskip

L'application était écrite en React avec l'utilisation de Redux. J'ai pu
donc mettre en application les principes appris lors de ma première
semaine.

\bigskip

Pour l'intégration du style du site, j'ai pu utiliser Zeplin. C'est une
application de collaboration pour les designers et les intégrateurs. Il
permet aux designers de télécharger leurs maquettes fonctionnelles
directement à partir de Sketch et les ajouter aux dossiers de projet
dans Zeplin. Les annotations seront automatiquement ajoutées aux designs
(tailles, couleurs, marges et même suggestions CSS pour certains
éléments). Il est alors beaucoup plus simple d'intégrer les maquettes.

\bigskip

Finalement, pour l'intégration des maquettes, j'ai fait le choix
d'utiliser SASS. Sass (Syntactically Awesome Stylesheets) est un langage
de génération dynamique de feuilles de style. On peut le voir comme une
extension de CSS3, ajoutant de nouvelles règles dans notre façon
d'intégrer un web design. Les principaux ajouts sont : les variables,
les mixins, l'héritage de sélection et différents options très utiles.

\bigskip

\subsubsection{Déroulement}\label{duxe9roulement}

\bigskip

\paragraph{Intégration des
maquettes}\label{intuxe9gration-des-maquettes}

\bigskip

La première partie du projet consisté à intégrer les maquettes fournit
par le client. Dans un premier temps, j'ai du redécouper l'application.
En effet, lors du premier jet réalisé par mon collègue, le client était
parti sur une application monopage. Mais à la réception des maquettes,
l'application était devenue multi-page. Il a donc tout d'abord fallut
mettre en place un routeur pour permettre à l'utilisateur de naviguer
entre les pages.

\bigskip

J'ai choisi, après quelques recherches, d'utiliser React Router.
\href{https://github.com/ReactTraining/react-router}{React Router} est
une bibliothèque de routage pour React. Il dispose d'une \emph{API}
simple avec des fonctionnalités puissantes. Il garde l'interface
utilisateur synchronisé avec l'\emph{URL}. React Router est très simple
à utiliser. Il suffit de lister les différentes routes souhaitées,
associées au composant correspondant.

\bigskip

Une fois ce découpage effectué, j'ai mis en place SASS pour la gestion
des feuilles de style. Pour faciliter son utilisation, j'ai créé
plusieurs fichiers avec les variables et fonctions (mixins) qui seront
utilisés dans tout le projet. Le fichier \texttt{variable} contient
notamment les codes hexadecimaux des couleurs de l'application, les
tailles des polices d'écriture utilisées, etc. Utiliser des variables
évite de devoir revenir sur tous les fichiers du projet si l'on décide
de changer l'une de ses variables.

\bigskip

Une fois ces fichiers SASS principaux créés, j'ai simplement créé un
fichier SASS pour chaque composant React de l'application, qui contient
donc tout le style de ce composant. J'ai du faire beaucoup de recherches
pour être à l'aise avec le CSS (Cascading Style Sheets), c'est à dire le
langage décrivant la présentation de l'application. En effet, jusque là
je n'avais pas eu l'occasion d'appronfondir mes connaissances en
intégration.

\bigskip

L'intégration du style fut assez rapide, l'application étant
visuellement assez simple. Grâce à cette première étape j'ai pu me
familiariser avec l'organisation du projet avant d'attaquer des parties
plus difficiles.

\bigskip

\paragraph{Formatage et téléchargement des
photos}\label{formatage-et-tuxe9luxe9chargement-des-photos}

\bigskip

La deuxième partie du projet était centrée sur le téléchargement des
photos vers le serveur. Le client a lui même développé une API assez
simple que nous devions manipuler pour envoyer les photos, changer le
nombre d'exemplaire, récupérer le prix de la commande\ldots{}

\bigskip

La majeur difficulté durant cette étape fut de ne pas avoir accès
directement à l'API. En effet, celle-ci évoluant en même temps que
l'application et ne possédant pas de documentation, il était souvent
nécessaire de demander des précisions ou des évolutions au client. Même
si la communication était assez rapide, le fait de ne pas avoir la main
sur l'API a ralenti le développement.

\bigskip

Une autre difficulté était le redimensionnement des images avant l'envoi
au serveur. En effet, quand l'utilisateur sélectionne des photos, nous
devons tout d'abord passer la photo au format sélectionné, et donc gérer
les formats incompatibles. Cela a apporté beaucoup de questions. Par
exemple, si une photo est en 10x15 et que l'utilisateur a sélectionné le
format 11x15 que fait-on ? Coupe-t-on des morceaux de la photo pour
arriver au format voulu ? Ou bien ajoute-t-on des bandes ?

\bigskip

Cela a donc donné lieu à beaucoup de discutions avec le client pour
résoudre toutes ses problèmatiques avant de developper les solutions. La
fonction de redimensionnement est une fonction clé du projet. Une fois
la photo redimensionnée, nous réduisons la résolution de la photo
jusqu'a 300dpi (point par pouce). C'est la résolution optimal pour
l'impression de photo : assez élevée pour garantir une bonne qualité à
l'impression, et assez faible pour rendre le téléchargement vers le
serveur le plus rapide possible.

\bigskip

\paragraph{Modification des photos}\label{modification-des-photos}

\bigskip

L'étape suivante est la création de l'interface et des fonctions
permettant à l'utilisateur de modifier ses photos. Il peut changer le
format, recadrer la photo, changer l'orientation\ldots{} J'ai tout
d'abord chercher s'il existait déja un outil pour le recadrage de la
photo.

\bigskip

C'est à ce moment que j'ai découvert la diversité des outils React
proposés par la communauté : il est trés facile de trouver des
composants sur Github qui correspondent à votre besoin. J'ai donc pu
utiliser
\href{https://github.com/roadmanfong/react-cropper}{react-cropper},
trouvé sur Github après quelques recherches, qui s'est révélé très
efficace. Il permet de gérer le recadrage, fournit des fonctions
renvoyant toutes les données intéressantes (dimensions du recadrage,
rotation de la zone de recadrage\ldots{}). Ce fut ma première
intégration d'un outil React.

\bigskip

Pour mettre en place mes autres fonctions de modification des photos,
j'ai surtout dû modifier la fonction principale de redimensionnement des
photos, utilisé lors du téléchargement initial. J'y ai ajouté des
paramétres permettant de choisir le format, l'orientation, si on coupe
la photo ou bien on ajoute des bandes blanche\ldots{}

\bigskip

Enfin, il a fallut mettre en place l'interface utilisateur. Ce fut une
étape

\paragraph{Pages informations des clients, paiement et
remerciement}\label{pages-informations-des-clients-paiement-et-remerciement}

\bigskip

La dernière partie du projet consisté à mettre en place les autres pages
de l'application, servant à récolter les informations nécessaires à la
commande, et à les envoyer à l'API. Ces pages étaient la page de saisie
des adresses, de livraison et de facturation, la page de paiement, soit
par carte bancaire, soit par Paypal, et enfin la page de remerciement
avec un récapitulatif de la commande, ainsi que des liens pour partager
l'événement sur les réseaux sociaux.

\bigskip

Pour la page de saisie de l'adresse, ce fut l'occasion pour moi de créer
pour la première fois un formulaire en React et Redux. Avec ce langage,
la création de formulaire est assez peut instactif car il faut
répercuter chaque changement des champs, chaque lettre écrite ou
effacée, pour que l'état de l'applications soit toujours à jour. C'est
assez fastidieux et inhabituel.

\bigskip

Après cette première expériences dans la création de formulaire, j'ai
pus découvrir un outil, redux-form, permettant de créer beaucoup plus
facilement des formulaires et gérant automatiquement la mise à jour de
l'état de l'application. J'ai pu utiliser cet outil dans mon second
projet. Mais je pense que le fait d'avoir d'abord du faire toute
l'implémentation nécéssaire par moi même m'a permis de prendre
conscience des problématiques de cette pile technologique : le maintiens
de l'état de l'application, la communication entre les
composants\ldots{}

\bigskip

J'ai ensuite travaillé sur le page de remerciement. Sur cette page, on
affiche un récapitulatif de la commande ainsi que des liens pour
partager sur les réseaux sociaux l'événement. J'ai ainsi pu apprendre
comment partager sur Facebook et Tweeter un message, associé a une URL.

\bigskip

Finalement, sur les trois pages précédemment citées, j'ai du faire
apparaitre un récapitulatif de la commande, avec notamment le nombre de
photos commandé, le prix par photos, le prix total, ainsi que la
possibilité d'entrer un code de promotion avant le paiement. Cette
partie à demandé de la réflection car le calculs des prix était
différent avant et après le paiement. En effet, dans l'absolue, il faut
récupérer le prix à partir de l'API, pour être certain de son
exactitude. Cependant, avant le paiement il est possible que toute les
photos ne soient pas encore envoyé à l'API et donc le pris renvoyé par
celle-ci n'est pas définitive. IL est alors donc nécéssaire de faire le
calcul du prix dans l'application, et d'afficher ce resultat dans le
récapitulatif. Une fois le paiement effectué, il faut afficher le prix
envoyé par l'API, puisqu'il s'agit du prix final.

\bigskip

\paragraph{Fin du projet}\label{fin-du-projet}

\bigskip

Vers mi-octobre, j'ai été réaffectée à un autre projet, laissant la fin
du projet Photolix à Fabien. Il a fini de mettre en place la gestion du
téléchargement des photos vers le serveur, notamment après leur
modification. Il a également revu la fonction de redimensionnement des
photos, car celle utilisée au debut n'était pas assez performante : si
l'on monté à une centaine de photos chargées, le navigateur ne
supportait pas la charge.

\bigskip

J'ai pu étudier les modifications apportées par Fabien, et apprendre des
erreurs que j'ai pu commettre. Par exemple, je n'avais pas assez
travailler la gestion des erreurs. Il a fallut que Fabien reprenne mon
travail pour ajouter l'affichage des erreurs, en cas par exemple
d'erreur lors du téléchargement des photos. Ces enseignements m'ont
permis de ne pas reproduire ces erreurs dans le second projet sur lequel
j'ai été affecté.

\bigskip

Finalement, j'ai pu retravailler sur le projet en decembre. Après un
premier rendu au client, celui-ci souhaité quelques corrections ainsi
que l'ajout de quelques fonctionnalités. Il avait utilisé un échantillon
de client pour tester l'application et avait relevé des améliorations
possibles à l'interface utilisateur. J'ai donc pu aider Fabien à mettre
en place ces modifications.

\bigskip

\subsubsection{Conclusion}\label{conclusion}

\bigskip

Ce premier projet chez Dernier cri m'a beaucoup apporté. Pour commencer,
j'ai pu me familiariser aves les méthodes de gestion de projet de
l'entreprise, le processus allant de la formalisation du probléme,
jusque sa mise en ligne. Dans le même temps j'ai pu prendre en main les
outils utiliser chez Dernier cri, tel que Github, avec toute la gestion
des différentes branches, les Pull Request ainsi que la gestion des
issues\ldots{} Ce fut une étape essenciel pour mon intégration dans
l'équipe de développement.

\bigskip

Ce projet m'a également permit d'apprendre à utiliser React et Redux.
Ces technologies sont aujourd'hui en pleine expensions, et évolue
beaucoup et ont un belle avenir devant elles. J'ai également pu
apprendre a deveopper en front-end, alors que jusque là mes compétences
techniques étaient plus tournée vers du back.

\bigskip

Evidemment, j'ai connus de nombreuse difficultés lors du développemment
de Photolix. Mon inexpérience m'a conduit à faire des erreurs, à fournir
un résultat qui n'était pas optimal. J'ai heureusement pu compter sur
mon collégue, Fabien Gavory, pour me soutenir, me corriger et rattraper
certaine erreurs, surtout vers la fin du projet. Grace à cela, j'ai pu
mettre en place de meilleurs pratiques durant le projet suivant.

\bigskip

Finalement, c'est une vraie chance d'avoir pu travailler dès mon arrivée
sur un projet pour un client. J'ai ainsi été tout de suite confrontée
aux vraie problématiques du developpement d'une application web, de la
relation cliente, avec des dates butoire et des évolutions des
spécifications en cours de projet.

\subsection{Finfrog}\label{finfrog}

\subsubsection{Présentation du
projet}\label{pruxe9sentation-du-projet-1}

\bigskip

Avec l'arrivée d'un nouveau client et la nécéssité de fournir un
developpeur React sur ce projet, j'ai quitté le projet Photolix pour
rejoindre Finfrog.

\bigskip

Finfrog est un projet de prêt collaboratif, c'est à dire que le site
propose des prêt financés par des particuliers. Les prêts proposés vont
de 200 à 600 euros, à rembourser en 1 à 3 mois. Le but de ce site est
d'ouvrir, en acceptant des prêts qui ne seraient pas validé par une
banque car ils sont trop faibles ou bien que la personne est au chômage.

\bigskip

Lors de mon arrivée sur le projet, un site été déjà en ligne, développé
par le client. Le premier objectif était de mettre en place un nouveau
design sur ce site, d'abord sur la page d'acceuil, et ensuite sur les
formulaires de demande de prêt.

\bigskip

Par la suite, j'ai été amené à developper de nouvelles fonctionnalités
pour Finfrog, comme la partie du site réservée à la gestion des prêts
par l'administrateur, les espaces emprunteur et préteur, la génération
de contrat.

\subsubsection{Objectifs}\label{objectifs-1}

\subsubsection{Nouveaux outils}\label{nouveaux-outils}

\bigskip

La projet Finfrog utilisait principalement les même outils que Photolix
: utilisation de npm pour la gestion des paquets, de Zeplin pour étudier
le design, de React et Redux etc.

\bigskip

Cependant à mon arrivée sur FinFrog, le projet était hébergé sur
Bitbucket et non pas Github. Il a donc fallut que je m'habitue à ce
nouveaux gestionaire. Par la suite nous avons migré le projet sur
Github.

\bigskip

Sur ce projet, nous avions aussi en charge la partie API et base de
donnée. L'API est écrite en Nodejs et la base de donnée est une
postgres, donc manipulable en SQL. Cependant il était rare que je doive
toucher à la base de donnée.

\bigskip

-\textgreater{} pm2r

\subsubsection{Déroulement}\label{duxe9roulement-1}

\begin{itemize}
\item
  Première partie design -\textgreater{} Page d'acceuil : prendre en
  main le projet et ses particuliarité -\textgreater{} Accés restraint
  au debut du projet -\textgreater{} Tunnel emprunteur : on rentre plus
  dans le techniques, la manipulation de redux
\item
  génération de contrat
\item
  Grosse passe pour le responsive + cross compatibilité des browsers
\item
  Communication avec le client ? -\textgreater{} Surtout semaine de noel
  sans Laetitia
\end{itemize}

Citer le fait que je n'ai presque pas eu de code review a cause de la
confidentialité

\subsubsection{Conclusion}\label{conclusion-1}

\newpage

\section{La vie Lilloise}\label{la-vie-lilloise}

Durant mon stage à Lille j'ai pu me rendre compte que la ville possédait
une communauté web très active.

\bigskip

La ville a notamment reçu le label `French Tech' fin 2014, pour
récompenser son dynamiste dans le numérique et l'innovation. Ce label,
en plus de récompenser les efforts de la ville, constitue le point
d'entrée vers des dispositifs nationaux comme des programmes pour
attirer les entrepreneurs étrangers qui veulent créer leur
\emph{start-up} en France.

\bigskip

La région profite de la présence de grands groupes nationaux comme
Orange, Capgemini, IBM France, CGI, CISCO\ldots{}

\bigskip

De plus Lille a mis en place un ensemble de structures favorisant
l'accompagnement et la croissance des startups vers un marché mondial.
La plus notable est évidemment Euratechnologie, le Pôle d'excellence
économique dédié aux Technologies de l'Information et de la
Communication (TIC) de la métropole lilloise. EuraTechnologies a été
classé dans le top 10 des accélérateurs de \emph{startup} d'Europe par
Fundacity, et le 1er en France. Euratechnologie posséde des espaces
dédiés à la recherche, la formation et l'entrepreneuriat, un incubateur
et un accélérateur.

\bigskip

La région lilloise posséde d'autre espaces dédié à l'innovation, les
\emph{startup} et l'entrepreneuriat : La Plaine Images à Tourcoing et
Roubaix, Eurasanté à Lille, La Haute Borne à Villeneuve d'Ascq, La Serre
Numérique à Valenciennes, Le Pôle Numérique Culturel Louvre Lens Vallée
de Lens\ldots{}

\bigskip

La région a connu l'émergence de nombreuse entreprises, prometteuses ou
déjà fructueuses : Big Ben, Ankama, OVH, Addictiz, Stereograph, Clic \&
Walk, Giroptic, Mazeberry, Vekia, Sparkow, Mdoloris, A-volute, Critizr,
Intent Technologies\ldots{}

\bigskip

Avoir eu la chance de faire mon stage dans cette région m'a permis de
profiter de cette écosysteme riche et actif. J'ai pu participer à des
conférences, des salons et des réunions qui m'ont beaucoup apporté, tant
au niveau technique que social. Cela m'a permit de préciser mon projet
professionnel, en m'immergeant dans la vie entreprenariale d'une ville.

\bigskip

\subsection{Take Off Conference}\label{take-off-conference}

\bigskip

Dernier cri m'a donné l'occasion durant mon stage d'assister à la
\href{http://takeoffconf.com/2016}{\textbf{Take Off Conference}} les 20
et 21 octobre 2016. Cet évènement a lieu depuis plusieurs années à
EuraTechnologies.

\bigskip

Historiquement, ce sont les fondateurs de Dernier cri qui ont créé la
\textbf{Take Off Conference}, avec Florian Le Goff. Aujourd'hui ce sont
d'autres acteurs de la communauté web de Lille qui ont pris le relais
pour proposer une nouvelle édition.

\bigskip

La \textbf{Take Off Conference} est un cycle de conférences anglophones.
L'événement dure 2 jours, et accueille des conférenciers du monde
entier. Bien qu'elle reste avant tout une conférence pour les
développeurs Web, elle reste accessible pour les développeurs en
général.

\bigskip

Dernier cri m'a permis de participer à cette conférence avec l'un de mes
collègues. Ce fut une véritable chance pour moi de rencontrer et
échanger avec des conférenciers du monde entier. Les conférences étaient
très intéréssantes et inspirantes, sur des sujets très variés allant de
la compréhension des enjeux de mise en place de nouveau outils, à des
sujets plus sociaux comme l'acceuil des developpeurs anglais après le
Brexit.

\bigskip

Des évènements ont également été organisé le soir pour permettre aux
participants et aux conférenciers d'échanger dans un cadre plus détendu.

\bigskip

Ce fut une excellente occasion de découvrir de nouvelles technologies,
de m'ouvrir à des problématiques que je ne connaissais pas ainsi que de
rencontrer des développeurs qualifiés et passionnés. Lors de ses
échanges, j'ai pu me rendre compte de la portée internationale de la
programmation : des personnes des quatres coins du monde se retrouvaient
sur les mêmes problématiques.

\bigskip

\subsection{Meetup}\label{meetup}

\bigskip

La communauté web de Lille est très active pour organisé des évenements.
De très nombreux Meetup sont organisés sur différents domaines du web,
acceuillant autant des experts que des débutants ou simplement des
curieux.

\bigskip

Très actif dans cette communauté, Dernier cri acceuille au sein de ses
locaux certaines de ses rencontres. J'ai notamment eu l'occasion
d'assister à des Meetup de Lille FP (Fonctionnal programming, c'est à
dire programmation fonctionnelle) et de Lille Elixir.

\bigskip

Certains membres de l'équipe sont également investis dans ces
rencontres, en tant qu'organisateur ou bien speaker. Les patrons les
poussent à (prendre part a la vie de la communauté web). J'ai ainsi pu
facilement être au courant des différents évenements Lillois et y
participer avec mes collégues, ce qui m'a permis d'être bien intégré.

\bigskip

\subsection{Maker Faire}\label{maker-faire}

\bigskip

La Maker Faire est un autre événement majeur organisé à Lille durant mon
stage. Ce concept totalement unique regroupe stands de démonstration,
ateliers de découverte, spectacles et conférences autour des thèmes de
la créativité, de la fabrication, des cultures Do It Yourself et Makers.

\bigskip

Cet événement, présenté par Leroy Merlin en partenariat avec la Ville de
Lille et lille3000 (programme culturel de la ville de Lille), réunit des
passionnés de technologies, des artisans, des industriels, des amateurs,
des ingénieurs, des clubs de science, des artistes, des étudiants et des
Start'Up. Ils forment la communauté des Makers et viennent pour montrer
leurs créations, partager leurs connaissances\ldots{}

\bigskip

Dernier cri a été invité à assister à la Maker Faire, notamment car nous
développons l'application web de TechShop, l'atelier collaboratif de
Leroy Merlin. Dans ce contexte j'ai pu découvrir la communauté
\emph{Maker} de Lille. J'ai pu notamment décrouvrir les \emph{repair
coffee}, lieux où l'on peut amener ses appareils électroniques cassés et
recevoir de l'aide pour leur réparation. Il y avait également des
robots, des imprimantes 3D, des casques de réalité augmentée \ldots{}
C'était un endroit plein d'aspiration et d'envie d'entreprendre.

\newpage

\section{Conclusion}\label{conclusion-2}

-\textgreater{} Compétences apprises : Node React Redux ES6 Relation
client Git Github pm2 npm

-\textgreater{} Important : relation client et autonomie, apprentissage
rapide, equipe disponible et polivalente (qui peut t'aider sur tout)

-\textgreater{} environnement cool : Ambiance, tech off conf , meetup,
plus belle vue de Lille

\newpage

\section{Glossaire ?}\label{glossaire}
